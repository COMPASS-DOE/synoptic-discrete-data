% Options for packages loaded elsewhere
\PassOptionsToPackage{unicode}{hyperref}
\PassOptionsToPackage{hyphens}{url}
%
\documentclass[
]{article}
\usepackage{amsmath,amssymb}
\usepackage{iftex}
\ifPDFTeX
  \usepackage[T1]{fontenc}
  \usepackage[utf8]{inputenc}
  \usepackage{textcomp} % provide euro and other symbols
\else % if luatex or xetex
  \usepackage{unicode-math} % this also loads fontspec
  \defaultfontfeatures{Scale=MatchLowercase}
  \defaultfontfeatures[\rmfamily]{Ligatures=TeX,Scale=1}
\fi
\usepackage{lmodern}
\ifPDFTeX\else
  % xetex/luatex font selection
\fi
% Use upquote if available, for straight quotes in verbatim environments
\IfFileExists{upquote.sty}{\usepackage{upquote}}{}
\IfFileExists{microtype.sty}{% use microtype if available
  \usepackage[]{microtype}
  \UseMicrotypeSet[protrusion]{basicmath} % disable protrusion for tt fonts
}{}
\makeatletter
\@ifundefined{KOMAClassName}{% if non-KOMA class
  \IfFileExists{parskip.sty}{%
    \usepackage{parskip}
  }{% else
    \setlength{\parindent}{0pt}
    \setlength{\parskip}{6pt plus 2pt minus 1pt}}
}{% if KOMA class
  \KOMAoptions{parskip=half}}
\makeatother
\usepackage{xcolor}
\usepackage[margin=1in]{geometry}
\usepackage{color}
\usepackage{fancyvrb}
\newcommand{\VerbBar}{|}
\newcommand{\VERB}{\Verb[commandchars=\\\{\}]}
\DefineVerbatimEnvironment{Highlighting}{Verbatim}{commandchars=\\\{\}}
% Add ',fontsize=\small' for more characters per line
\usepackage{framed}
\definecolor{shadecolor}{RGB}{248,248,248}
\newenvironment{Shaded}{\begin{snugshade}}{\end{snugshade}}
\newcommand{\AlertTok}[1]{\textcolor[rgb]{0.94,0.16,0.16}{#1}}
\newcommand{\AnnotationTok}[1]{\textcolor[rgb]{0.56,0.35,0.01}{\textbf{\textit{#1}}}}
\newcommand{\AttributeTok}[1]{\textcolor[rgb]{0.13,0.29,0.53}{#1}}
\newcommand{\BaseNTok}[1]{\textcolor[rgb]{0.00,0.00,0.81}{#1}}
\newcommand{\BuiltInTok}[1]{#1}
\newcommand{\CharTok}[1]{\textcolor[rgb]{0.31,0.60,0.02}{#1}}
\newcommand{\CommentTok}[1]{\textcolor[rgb]{0.56,0.35,0.01}{\textit{#1}}}
\newcommand{\CommentVarTok}[1]{\textcolor[rgb]{0.56,0.35,0.01}{\textbf{\textit{#1}}}}
\newcommand{\ConstantTok}[1]{\textcolor[rgb]{0.56,0.35,0.01}{#1}}
\newcommand{\ControlFlowTok}[1]{\textcolor[rgb]{0.13,0.29,0.53}{\textbf{#1}}}
\newcommand{\DataTypeTok}[1]{\textcolor[rgb]{0.13,0.29,0.53}{#1}}
\newcommand{\DecValTok}[1]{\textcolor[rgb]{0.00,0.00,0.81}{#1}}
\newcommand{\DocumentationTok}[1]{\textcolor[rgb]{0.56,0.35,0.01}{\textbf{\textit{#1}}}}
\newcommand{\ErrorTok}[1]{\textcolor[rgb]{0.64,0.00,0.00}{\textbf{#1}}}
\newcommand{\ExtensionTok}[1]{#1}
\newcommand{\FloatTok}[1]{\textcolor[rgb]{0.00,0.00,0.81}{#1}}
\newcommand{\FunctionTok}[1]{\textcolor[rgb]{0.13,0.29,0.53}{\textbf{#1}}}
\newcommand{\ImportTok}[1]{#1}
\newcommand{\InformationTok}[1]{\textcolor[rgb]{0.56,0.35,0.01}{\textbf{\textit{#1}}}}
\newcommand{\KeywordTok}[1]{\textcolor[rgb]{0.13,0.29,0.53}{\textbf{#1}}}
\newcommand{\NormalTok}[1]{#1}
\newcommand{\OperatorTok}[1]{\textcolor[rgb]{0.81,0.36,0.00}{\textbf{#1}}}
\newcommand{\OtherTok}[1]{\textcolor[rgb]{0.56,0.35,0.01}{#1}}
\newcommand{\PreprocessorTok}[1]{\textcolor[rgb]{0.56,0.35,0.01}{\textit{#1}}}
\newcommand{\RegionMarkerTok}[1]{#1}
\newcommand{\SpecialCharTok}[1]{\textcolor[rgb]{0.81,0.36,0.00}{\textbf{#1}}}
\newcommand{\SpecialStringTok}[1]{\textcolor[rgb]{0.31,0.60,0.02}{#1}}
\newcommand{\StringTok}[1]{\textcolor[rgb]{0.31,0.60,0.02}{#1}}
\newcommand{\VariableTok}[1]{\textcolor[rgb]{0.00,0.00,0.00}{#1}}
\newcommand{\VerbatimStringTok}[1]{\textcolor[rgb]{0.31,0.60,0.02}{#1}}
\newcommand{\WarningTok}[1]{\textcolor[rgb]{0.56,0.35,0.01}{\textbf{\textit{#1}}}}
\usepackage{graphicx}
\makeatletter
\def\maxwidth{\ifdim\Gin@nat@width>\linewidth\linewidth\else\Gin@nat@width\fi}
\def\maxheight{\ifdim\Gin@nat@height>\textheight\textheight\else\Gin@nat@height\fi}
\makeatother
% Scale images if necessary, so that they will not overflow the page
% margins by default, and it is still possible to overwrite the defaults
% using explicit options in \includegraphics[width, height, ...]{}
\setkeys{Gin}{width=\maxwidth,height=\maxheight,keepaspectratio}
% Set default figure placement to htbp
\makeatletter
\def\fps@figure{htbp}
\makeatother
\setlength{\emergencystretch}{3em} % prevent overfull lines
\providecommand{\tightlist}{%
  \setlength{\itemsep}{0pt}\setlength{\parskip}{0pt}}
\setcounter{secnumdepth}{-\maxdimen} % remove section numbering
\ifLuaTeX
  \usepackage{selnolig}  % disable illegal ligatures
\fi
\usepackage{bookmark}
\IfFileExists{xurl.sty}{\usepackage{xurl}}{} % add URL line breaks if available
\urlstyle{same}
\hypersetup{
  pdftitle={Synoptic\_CB\_Nutrients\_2023\_AnalysisTemplate},
  pdfauthor={Month of Data Being Analyzed},
  hidelinks,
  pdfcreator={LaTeX via pandoc}}

\title{Synoptic\_CB\_Nutrients\_2023\_AnalysisTemplate}
\author{Month of Data Being Analyzed}
\date{2025-06-23}

\begin{document}
\maketitle

\#\#Run Information

\begin{Shaded}
\begin{Highlighting}[]
\CommentTok{\#let you know which section you are in }
\FunctionTok{cat}\NormalTok{(}\StringTok{"Run Information: Input by User"}\NormalTok{)}
\end{Highlighting}
\end{Shaded}

\begin{verbatim}
## Run Information: Input by User
\end{verbatim}

\begin{Shaded}
\begin{Highlighting}[]
\CommentTok{\#set the run date \& user name }
\NormalTok{  run\_date }\OtherTok{\textless{}{-}} \StringTok{"20240114"}
\NormalTok{  sample\_year }\OtherTok{\textless{}{-}} \DecValTok{2023}
\NormalTok{  sample\_month }\OtherTok{\textless{}{-}} \DecValTok{06}
\NormalTok{  user }\OtherTok{\textless{}{-}} \StringTok{"Stephanie Wilson"}

\CommentTok{\#identify the files you want to read in }
  \CommentTok{\#read in as a list to accommadate ultiple runs in a month}
\NormalTok{  NOx\_files }\OtherTok{\textless{}{-}} \FunctionTok{c}\NormalTok{(}\StringTok{"Raw Data/SEAL\_COMPASS\_Synoptic\_NOx\_June2023\_1.csv"}\NormalTok{,}
                 \StringTok{"Raw Data/SEAL\_COMPASS\_Synoptic\_NOx\_June2023\_2.csv"}\NormalTok{,}
                 \StringTok{"Raw Data/SEAL\_COMPASS\_Synoptic\_NOx\_June2023\_3.csv"}\NormalTok{)}
\NormalTok{  NH3\_PO4\_files }\OtherTok{\textless{}{-}} \FunctionTok{c}\NormalTok{(}\StringTok{"Raw Data/SEAL\_COMPASS\_Synoptic\_NH3\_PO4\_June2023\_1.csv"}\NormalTok{, }
                     \StringTok{"Raw Data/SEAL\_COMPASS\_Synoptic\_NH3\_PO4\_June2023\_2.csv"}\NormalTok{,}
                     \StringTok{"Raw Data/SEAL\_COMPASS\_Synoptic\_NH3\_PO4\_June2023\_3.csv"}\NormalTok{)}

\CommentTok{\# Define the file path for QAQC log file {-} NO Need to change just check year }
\NormalTok{  file\_path }\OtherTok{\textless{}{-}} \StringTok{"Raw Data/SEAL\_COMPASS\_Synoptic\_QAQC\_Log\_2023.csv"}
\NormalTok{  final\_path }\OtherTok{\textless{}{-}} \StringTok{"Processed Data/COMPASS\_Synoptic\_Nutrients\_202306.csv"}

\CommentTok{\#record any notes about the run or anything other info here: }
\NormalTok{  run\_notes }\OtherTok{\textless{}{-}} \StringTok{"There are two sample names we suspect were input incorrectly,}
\StringTok{  they are listed below and have been checked against metadata. The metadata from Goodwin ans Sweethall samples is not present."}
  
\CommentTok{\#duplicate sample names to be changed }
  \CommentTok{\#list the sample iDs that are messed up and create a list }
  \CommentTok{\#with run number as well so that we can change them below }
\NormalTok{  wrong\_names }\OtherTok{\textless{}{-}} \FunctionTok{c}\NormalTok{(}\StringTok{"GCW\_202304\_TR\_LysC\_45cm"}\NormalTok{, }\StringTok{"GCW\_202304\_TR\_LysA\_20cm\_8"}\NormalTok{,}
                   \StringTok{"GWI\_202304\_UP\_LysA\_20cm"}\NormalTok{, }\StringTok{"GWI\_202304\_UP\_LysA\_20cm"}\NormalTok{)}
\NormalTok{  wrong\_nums }\OtherTok{\textless{}{-}} \FunctionTok{c}\NormalTok{(}\DecValTok{20}\NormalTok{, }\DecValTok{16}\NormalTok{, }\DecValTok{46}\NormalTok{, }\DecValTok{44}\NormalTok{)}
\NormalTok{  correct\_names }\OtherTok{\textless{}{-}} \FunctionTok{c}\NormalTok{(}\StringTok{"GCW\_202304\_TR\_LysB\_45cm"}\NormalTok{, }\StringTok{"GCW\_202304\_TR\_LysA\_20cm"}\NormalTok{,}
                     \StringTok{"GWI\_202304\_UP\_LysA\_10cm"}\NormalTok{, }\StringTok{"GWI\_202304\_UP\_LysA\_10cm"}\NormalTok{)}
  
  \CommentTok{\#can\textquotesingle{}t determine from metadata {-} for now unsure }
\NormalTok{  remove\_names }\OtherTok{\textless{}{-}} \FunctionTok{c}\NormalTok{(}\StringTok{"GCW\_202304\_TR\_LysA\_20cm"}\NormalTok{, }\StringTok{"GCW\_202304\_TR\_LysA\_20cm"}\NormalTok{,}
                    \StringTok{"GCW\_202304\_TR\_LysB\_20cm\_13"}\NormalTok{, }\StringTok{"GCW\_202304\_TR\_LysB\_20cm\_13"}\NormalTok{) }
      \CommentTok{\#couldn\textquotesingle{}t tell which onethis is from the metadata, no A\_10cm which is what we thought}
      \CommentTok{\#marked on the sheet, need to check sample vials in freezer }
      \CommentTok{\#to see if we have a A\_10cm from GCW\_TR to be sure }
\NormalTok{  remove\_nums }\OtherTok{\textless{}{-}} \FunctionTok{c}\NormalTok{(}\DecValTok{15}\NormalTok{, }\DecValTok{13}\NormalTok{, }\DecValTok{21}\NormalTok{, }\DecValTok{19}\NormalTok{ )}
  
\CommentTok{\#Set up file path for metadata }
  \CommentTok{\#downloaded metadata csv {-} downloaded from Google drive as csv for this year}
  \CommentTok{\#https://docs.google.com/spreadsheets/d/1HCAN0\_q6y17x0RUXVzID09hVal{-}RfwWc/edit?usp=sharing\&ouid=108994740386869376571\&rtpof=true\&sd=true}
\NormalTok{  Raw\_Metadata }\OtherTok{=} \StringTok{"Raw Data/COMPASS\_SynopticCB\_PW\_SampleLog\_2023.csv"}
\end{Highlighting}
\end{Shaded}

\#\#Setup

\#\#Read in metadata and create similar sample IDs for matching to
samples

\subsection{Import Data \& Clean}\label{import-data-clean}

\subsection{Assessing standard Curves}\label{assessing-standard-curves}

\#Pull out standards data

\begin{verbatim}
## Assess Standard Curves
\end{verbatim}

\#Plot standards data

\begin{verbatim}
## Assess Standard Curves
\end{verbatim}

\includegraphics{COMPASS_Synoptic_NUTR_Analysis_TEMPLATE_2023_files/figure-latex/Assess Standard Curves-1.pdf}

\begin{verbatim}
## `geom_smooth()` using formula = 'y ~ x'
\end{verbatim}

\includegraphics{COMPASS_Synoptic_NUTR_Analysis_TEMPLATE_2023_files/figure-latex/Assess Standard Curves-2.pdf}

\begin{verbatim}
## `geom_smooth()` using formula = 'y ~ x'
\end{verbatim}

\includegraphics{COMPASS_Synoptic_NUTR_Analysis_TEMPLATE_2023_files/figure-latex/Assess Standard Curves-3.pdf}

\begin{verbatim}
## [1] "NOx Curve r2 GOOD - PROCEED"
\end{verbatim}

\begin{verbatim}
## [1] "NH3 Curve r2 GOOD - PROCEED"
\end{verbatim}

\begin{verbatim}
## [1] "PO4 Curve r2 GOOD - PROCEED"
\end{verbatim}

\begin{verbatim}
## [1] "QAQC log file exists and has been read into the code."
\end{verbatim}

\includegraphics{COMPASS_Synoptic_NUTR_Analysis_TEMPLATE_2023_files/figure-latex/Assess Standard Curves-4.pdf}

\begin{verbatim}
## # A tibble: 3 x 2
##   Test  avg_slope
##   <chr>     <dbl>
## 1 NH3        1.91
## 2 NOx        1.25
## 3 PO4        2.32
\end{verbatim}

\newpage

\subsection{Dilution Corrections - ensure the latest dilution is
kept}\label{dilution-corrections---ensure-the-latest-dilution-is-kept}

\begin{verbatim}
## Dilution Corrections
\end{verbatim}

\begin{verbatim}
## Duplicated samples: MSM_202306_TR_LysB_20cm, GCW_202306_WC_LysA_45cm, SWH_202306_TR_LysB_45cm, SWH_202306_TR_LysA_20cm
\end{verbatim}

\begin{verbatim}
## Dilution Present, Need to Correct
\end{verbatim}

\#Check NOx Reduction Efficiency

\begin{verbatim}
## Assess Reduction Efficiency
\end{verbatim}

\includegraphics{COMPASS_Synoptic_NUTR_Analysis_TEMPLATE_2023_files/figure-latex/Assess reduction Efficiency-1.pdf}

\begin{verbatim}
## [1] "Mean NOx Reduction Efficiency <95% - REASSESS"
\end{verbatim}

\begin{verbatim}
## [1] 94.7757
\end{verbatim}

\newpage

\subsection{Analyze the Check
Standards}\label{analyze-the-check-standards}

\begin{verbatim}
## Analyze Check Standards
\end{verbatim}

\begin{verbatim}
## [1] "NOx Check Standard RSD within Range - PROCEED"
\end{verbatim}

\begin{verbatim}
## [1] "NH3 Check Standard RSD within Range - PROCEED"
\end{verbatim}

\begin{verbatim}
## [1] "PO4 Check Standard RSD within Range - PROCEED"
\end{verbatim}

\includegraphics{COMPASS_Synoptic_NUTR_Analysis_TEMPLATE_2023_files/figure-latex/Check Standards-1.pdf}

\begin{verbatim}
## [1] ">60% of NOx Check Standards are within range of expected concentration - PROCEED"
\end{verbatim}

\begin{verbatim}
## [1] ">60% of NH3 Check Standards are within range of expected concentration - PROCEED"
\end{verbatim}

\begin{verbatim}
## [1] ">60% of PO4 Check Standards are within range of expected concentration - PROCEED"
\end{verbatim}

\newpage

\subsection{Analyze Blanks}\label{analyze-blanks}

\begin{verbatim}
## Assess Blanks
\end{verbatim}

\begin{verbatim}
## [1] ">60% of NOx Blank concentrations are lower than the lower 25% quartile of samples - PROCEED"
\end{verbatim}

\begin{verbatim}
## [1] ">60% of NH3 Blank concentrations are lower than the lower 25% quartile of samples - PROCEED"
\end{verbatim}

\begin{verbatim}
## [1] ">60% of PO4 Blank concentrations are lower than the lower 25% quartile of samples- PROCEED"
\end{verbatim}

\includegraphics{COMPASS_Synoptic_NUTR_Analysis_TEMPLATE_2023_files/figure-latex/Analyze Blanks-1.pdf}

\begin{verbatim}
## NOx blanks:
\end{verbatim}

\begin{verbatim}
## [1] 0.002287211
\end{verbatim}

\begin{verbatim}
## NH3 blanks:
\end{verbatim}

\begin{verbatim}
## [1] -0.008405619
\end{verbatim}

\begin{verbatim}
## PO4 blanks:
\end{verbatim}

\begin{verbatim}
## [1] 0.002676952
\end{verbatim}

\newpage

\subsection{Analyze Duplicates}\label{analyze-duplicates}

\begin{verbatim}
## Analyze Duplicates
\end{verbatim}

\begin{verbatim}
## [1] "<60% of NOx Duplicates have a CV <10% - REASSESS"
\end{verbatim}

\begin{verbatim}
## [1] ">60% of NH3 Duplicates have a CV <10% - PROCEED"
\end{verbatim}

\begin{verbatim}
## [1] ">60% of PO4 Duplicates have a CV <10% - PROCEED"
\end{verbatim}

\begin{verbatim}
## Warning: Using `size` aesthetic for lines was deprecated in ggplot2 3.4.0.
## i Please use `linewidth` instead.
## This warning is displayed once every 8 hours.
## Call `lifecycle::last_lifecycle_warnings()` to see where this warning was
## generated.
\end{verbatim}

\includegraphics{COMPASS_Synoptic_NUTR_Analysis_TEMPLATE_2023_files/figure-latex/Analyze Duplicates-1.pdf}

\newpage

\subsection{Spikes}\label{spikes}

\begin{verbatim}
## [1] ">60% of Spikes have a CV <50% - PROCEED"
\end{verbatim}

\begin{verbatim}
## [1] ">60% of Spikes have a CV <50% - PROCEED"
\end{verbatim}

\begin{verbatim}
## [1] ">60% of Spikes have a CV <50% - PROCEED"
\end{verbatim}

\includegraphics{COMPASS_Synoptic_NUTR_Analysis_TEMPLATE_2023_files/figure-latex/Analyze Spikes-1.pdf}

\newpage

\subsection{Matrix Effects}\label{matrix-effects}

\begin{verbatim}
## [1] "NO NOx Matrix Effect, PROCEED"
\end{verbatim}

\begin{verbatim}
## [1] "NO NH3 Matrix Effect, PROCEED"
\end{verbatim}

\begin{verbatim}
## [1] "NO PO4 Matrix Effect, PROCEED"
\end{verbatim}

\subsection{Add Unit Converted Data Column (mg/L to uM
)}\label{add-unit-converted-data-column-mgl-to-um}

\subsection{Sample Flagging - Within range of standard
curve}\label{sample-flagging---within-range-of-standard-curve}

\begin{verbatim}
## Sample Flagging
\end{verbatim}

\subsection{Pull out sample id
information}\label{pull-out-sample-id-information}

\begin{verbatim}
## Sample Processing
\end{verbatim}

\begin{verbatim}
## Warning: Expected 5 pieces. Missing pieces filled with `NA` in 36 rows [22, 23, 24, 49,
## 50, 51, 87, 88, 89, 115, 116, 117, 139, 140, 141, 181, 182, 183, 209, 210,
## ...].
\end{verbatim}

\begin{verbatim}
## Warning: There was 1 warning in `mutate()`.
## i In argument: `Samp_Time = ym(Samp_Time)`.
## Caused by warning:
## !  2 failed to parse.
\end{verbatim}

\subsection{Pulling Rhizon Samples}\label{pulling-rhizon-samples}

\begin{Shaded}
\begin{Highlighting}[]
\FunctionTok{cat}\NormalTok{(}\StringTok{"Rhizon Samples"}\NormalTok{)}
\end{Highlighting}
\end{Shaded}

\begin{verbatim}
## Rhizon Samples
\end{verbatim}

\begin{Shaded}
\begin{Highlighting}[]
\CommentTok{\# Filter rhizon and peeper samples}
\NormalTok{df\_rhizon }\OtherTok{\textless{}{-}}\NormalTok{ df\_all }\SpecialCharTok{\%\textgreater{}\%}
  \FunctionTok{filter}\NormalTok{(}\FunctionTok{str\_detect}\NormalTok{(Sample\_Name, }\StringTok{"RHZ"}\NormalTok{))}
\NormalTok{df\_peep }\OtherTok{\textless{}{-}}\NormalTok{ df\_all }\SpecialCharTok{\%\textgreater{}\%}
  \FunctionTok{filter}\NormalTok{(}\FunctionTok{str\_detect}\NormalTok{(Sample\_Name, }\StringTok{"PPR"}\NormalTok{))}

\CommentTok{\# Timestamp for backups}
\NormalTok{timestamp }\OtherTok{\textless{}{-}} \FunctionTok{format}\NormalTok{(}\FunctionTok{Sys.time}\NormalTok{(), }\StringTok{"\%Y{-}\%m{-}\%d\_\%H\%M"}\NormalTok{)}

\CommentTok{\# Paths}
\NormalTok{folder\_path }\OtherTok{\textless{}{-}} \FunctionTok{file.path}\NormalTok{(}\StringTok{"Raw Data"}\NormalTok{, }\StringTok{"Rhizon+Peeper"}\NormalTok{)}
\FunctionTok{dir.create}\NormalTok{(folder\_path, }\AttributeTok{recursive =} \ConstantTok{TRUE}\NormalTok{, }\AttributeTok{showWarnings =} \ConstantTok{FALSE}\NormalTok{)}

\NormalTok{rhizon\_main }\OtherTok{\textless{}{-}} \FunctionTok{file.path}\NormalTok{(folder\_path, }\StringTok{"rhizon\_data.csv"}\NormalTok{)}
\NormalTok{peeper\_main }\OtherTok{\textless{}{-}} \FunctionTok{file.path}\NormalTok{(folder\_path, }\StringTok{"peeper\_data.csv"}\NormalTok{)}

\NormalTok{rhizon\_backup }\OtherTok{\textless{}{-}} \FunctionTok{file.path}\NormalTok{(folder\_path, }\FunctionTok{paste0}\NormalTok{(}\StringTok{"rhizon\_data\_"}\NormalTok{, timestamp, }\StringTok{".csv"}\NormalTok{))}
\NormalTok{peeper\_backup }\OtherTok{\textless{}{-}} \FunctionTok{file.path}\NormalTok{(folder\_path, }\FunctionTok{paste0}\NormalTok{(}\StringTok{"peeper\_data\_"}\NormalTok{, timestamp, }\StringTok{".csv"}\NormalTok{))}

\CommentTok{\# Write timestamped backups}
\FunctionTok{write.csv}\NormalTok{(df\_rhizon, rhizon\_backup, }\AttributeTok{row.names =} \ConstantTok{FALSE}\NormalTok{)}
\FunctionTok{write.csv}\NormalTok{(df\_peep, peeper\_backup, }\AttributeTok{row.names =} \ConstantTok{FALSE}\NormalTok{)}

\CommentTok{\# Overwrite the main files with latest data}
\FunctionTok{write.csv}\NormalTok{(df\_rhizon, rhizon\_main, }\AttributeTok{row.names =} \ConstantTok{FALSE}\NormalTok{)}
\FunctionTok{write.csv}\NormalTok{(df\_peep, peeper\_main, }\AttributeTok{row.names =} \ConstantTok{FALSE}\NormalTok{)}

\DocumentationTok{\#\# \^{}\^{} I think there is a cleaner way to write this out, but this should work for now \^{}\^{}}
\end{Highlighting}
\end{Shaded}

\subsection{Check to see if samples run match metadata \& merge
info}\label{check-to-see-if-samples-run-match-metadata-merge-info}

\begin{verbatim}
## Check Sample IDs with Metadata
\end{verbatim}

\begin{verbatim}
## Some sample IDs are missing from metadata.
\end{verbatim}

\begin{verbatim}
##  [1] "MSM_202306_UP_LysA_45cm"    "MSM_202306_UP_LysC_20cm"   
##  [3] "MSM_202306_UP_LysC_45cm"    "MSM_202306_TR_LysA_45cm"   
##  [5] "SWH_202306_UPCON_LysA_10cm" "SWH_202306_UPCON_LysA_20cm"
##  [7] "SWH_202306_UPCON_LysA_45cm" "SWH_202306_UPCON_LysB_10cm"
##  [9] "SWH_202306_UPCON_LysB_20cm" "SWH_202306_UPCON_LysB_45cm"
## [11] "SWH_202306_UPCON_LysC_10cm" "SWH_202306_UPCON_LysC_20cm"
## [13] "SWH_202306_UPCON_LysC_45cm" "SWH_202306_UP_LysA_10cm"   
## [15] "SWH_202306_UP_LysA_20cm"    "SWH_202306_UP_LysA_45cm"   
## [17] "SWH_202306_UP_LysB_10cm"    "SWH_202306_UP_LysB_20cm"   
## [19] "SWH_202306_UP_LysB_45cm"    "SWH_202306_UP_LysC_10cm"   
## [21] "SWH_202306_UP_LysC_20cm"    "SWH_202306_UP_LysC_45cm"   
## [23] "SWH_202306_TR_LysA_10cm"    "SWH_202306_TR_LysA_20cm"   
## [25] "SWH_202306_TR_LysA_45cm"    "SWH_202306_TR_LysB_10cm"   
## [27] "SWH_202306_TR_LysB_20cm"    "SWH_202306_TR_LysB_45cm"   
## [29] "SWH_202306_TR_LysC_10cm"    "SWH_202306_TR_LysC_20cm"   
## [31] "SWH_202306_TR_LysC_45cm"    "SWH_202306_WC_LysA_10cm"   
## [33] "SWH_202306_WC_LysA_45cm"    "SWH_202306_WC_LysB_10cm"   
## [35] "SWH_202306_WC_LysB_20cm"    "SWH_202306_WC_LysB_45cm"   
## [37] "SWH_202306_WC_LysC_10cm"    "SWH_202306_WC_LysC_20cm"   
## [39] "SWH_202306_WC_LysC_45cm"    "SWH_202306_SW_A"           
## [41] "SWH_202306_SW_B"            "SWH_202306_SW_C"           
## [43] "GWI_202306_UP_LysA_10cm"    "GWI_202306_UP_LysA_20cm"   
## [45] "GWI_202306_UP_LysA_45cm"    "GWI_202306_UP_LysB_10cm"   
## [47] "GWI_202306_UP_LysB_20cm"    "GWI_202306_UP_LysB_45cm"   
## [49] "GWI_202306_UP_LysC_10cm"    "GWI_202306_UP_LysC_20cm"   
## [51] "GWI_202306_UP_LysC_45cm"    "GWI_202306_TR_LysA_10cm"   
## [53] "GWI_202306_TR_LysA_20cm"    "GWI_202306_TR_LysB_10cm"   
## [55] "GWI_202306_TR_LysB_20cm"    "GWI_202306_TR_LysB_45cm"   
## [57] "GWI_202306_TR_LysC_10cm"    "GWI_202306_TR_LysC_20cm"   
## [59] "GWI_202306_TR_LysC_45cm"    "GWI_202306_WC_LysA_10cm"   
## [61] "GWI_202306_WC_LysA_20cm"    "GWI_202306_WC_LysA_45cm"   
## [63] "GWI_202306_WC_LysB_10cm"    "GWI_202306_WC_LysB_20cm"   
## [65] "GWI_202306_WC_LysC_10cm"    "GWI_202306_WC_LysC_20cm"   
## [67] "GWI_202306_WC_LysB_45cm"    "GWI_202306_SW_A"           
## [69] "GWI_202306_SW_B"            "GWI_202306_SW_C"           
## [71] "MSM_202036_WC_LysB_10cm"    "GWI_202306_WC_LysC_45cm"
\end{verbatim}

\begin{verbatim}
## Warning: Expected 5 pieces. Missing pieces filled with `NA` in 36 rows [22, 23, 24, 49,
## 50, 51, 87, 88, 89, 115, 116, 117, 139, 140, 141, 181, 182, 183, 209, 210,
## ...].
\end{verbatim}

\begin{verbatim}
## Warning: There was 1 warning in `mutate()`.
## i In argument: `Samp_Time = ym(Samp_Time)`.
## Caused by warning:
## !  2 failed to parse.
\end{verbatim}

\newpage

\subsection{Visualize Data}\label{visualize-data}

\begin{verbatim}
## Visualize Data
\end{verbatim}

\includegraphics{COMPASS_Synoptic_NUTR_Analysis_TEMPLATE_2023_files/figure-latex/Visualize Data-1.pdf}
\includegraphics{COMPASS_Synoptic_NUTR_Analysis_TEMPLATE_2023_files/figure-latex/Visualize Data-2.pdf}
\includegraphics{COMPASS_Synoptic_NUTR_Analysis_TEMPLATE_2023_files/figure-latex/Visualize Data-3.pdf}

\subsection{Export Processed Data}\label{export-processed-data}

\#end

\end{document}
