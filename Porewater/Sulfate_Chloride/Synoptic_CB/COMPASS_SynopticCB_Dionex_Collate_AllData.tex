% Options for packages loaded elsewhere
\PassOptionsToPackage{unicode}{hyperref}
\PassOptionsToPackage{hyphens}{url}
%
\documentclass[
]{article}
\usepackage{amsmath,amssymb}
\usepackage{iftex}
\ifPDFTeX
  \usepackage[T1]{fontenc}
  \usepackage[utf8]{inputenc}
  \usepackage{textcomp} % provide euro and other symbols
\else % if luatex or xetex
  \usepackage{unicode-math} % this also loads fontspec
  \defaultfontfeatures{Scale=MatchLowercase}
  \defaultfontfeatures[\rmfamily]{Ligatures=TeX,Scale=1}
\fi
\usepackage{lmodern}
\ifPDFTeX\else
  % xetex/luatex font selection
\fi
% Use upquote if available, for straight quotes in verbatim environments
\IfFileExists{upquote.sty}{\usepackage{upquote}}{}
\IfFileExists{microtype.sty}{% use microtype if available
  \usepackage[]{microtype}
  \UseMicrotypeSet[protrusion]{basicmath} % disable protrusion for tt fonts
}{}
\makeatletter
\@ifundefined{KOMAClassName}{% if non-KOMA class
  \IfFileExists{parskip.sty}{%
    \usepackage{parskip}
  }{% else
    \setlength{\parindent}{0pt}
    \setlength{\parskip}{6pt plus 2pt minus 1pt}}
}{% if KOMA class
  \KOMAoptions{parskip=half}}
\makeatother
\usepackage{xcolor}
\usepackage[margin=1in]{geometry}
\usepackage{color}
\usepackage{fancyvrb}
\newcommand{\VerbBar}{|}
\newcommand{\VERB}{\Verb[commandchars=\\\{\}]}
\DefineVerbatimEnvironment{Highlighting}{Verbatim}{commandchars=\\\{\}}
% Add ',fontsize=\small' for more characters per line
\usepackage{framed}
\definecolor{shadecolor}{RGB}{248,248,248}
\newenvironment{Shaded}{\begin{snugshade}}{\end{snugshade}}
\newcommand{\AlertTok}[1]{\textcolor[rgb]{0.94,0.16,0.16}{#1}}
\newcommand{\AnnotationTok}[1]{\textcolor[rgb]{0.56,0.35,0.01}{\textbf{\textit{#1}}}}
\newcommand{\AttributeTok}[1]{\textcolor[rgb]{0.13,0.29,0.53}{#1}}
\newcommand{\BaseNTok}[1]{\textcolor[rgb]{0.00,0.00,0.81}{#1}}
\newcommand{\BuiltInTok}[1]{#1}
\newcommand{\CharTok}[1]{\textcolor[rgb]{0.31,0.60,0.02}{#1}}
\newcommand{\CommentTok}[1]{\textcolor[rgb]{0.56,0.35,0.01}{\textit{#1}}}
\newcommand{\CommentVarTok}[1]{\textcolor[rgb]{0.56,0.35,0.01}{\textbf{\textit{#1}}}}
\newcommand{\ConstantTok}[1]{\textcolor[rgb]{0.56,0.35,0.01}{#1}}
\newcommand{\ControlFlowTok}[1]{\textcolor[rgb]{0.13,0.29,0.53}{\textbf{#1}}}
\newcommand{\DataTypeTok}[1]{\textcolor[rgb]{0.13,0.29,0.53}{#1}}
\newcommand{\DecValTok}[1]{\textcolor[rgb]{0.00,0.00,0.81}{#1}}
\newcommand{\DocumentationTok}[1]{\textcolor[rgb]{0.56,0.35,0.01}{\textbf{\textit{#1}}}}
\newcommand{\ErrorTok}[1]{\textcolor[rgb]{0.64,0.00,0.00}{\textbf{#1}}}
\newcommand{\ExtensionTok}[1]{#1}
\newcommand{\FloatTok}[1]{\textcolor[rgb]{0.00,0.00,0.81}{#1}}
\newcommand{\FunctionTok}[1]{\textcolor[rgb]{0.13,0.29,0.53}{\textbf{#1}}}
\newcommand{\ImportTok}[1]{#1}
\newcommand{\InformationTok}[1]{\textcolor[rgb]{0.56,0.35,0.01}{\textbf{\textit{#1}}}}
\newcommand{\KeywordTok}[1]{\textcolor[rgb]{0.13,0.29,0.53}{\textbf{#1}}}
\newcommand{\NormalTok}[1]{#1}
\newcommand{\OperatorTok}[1]{\textcolor[rgb]{0.81,0.36,0.00}{\textbf{#1}}}
\newcommand{\OtherTok}[1]{\textcolor[rgb]{0.56,0.35,0.01}{#1}}
\newcommand{\PreprocessorTok}[1]{\textcolor[rgb]{0.56,0.35,0.01}{\textit{#1}}}
\newcommand{\RegionMarkerTok}[1]{#1}
\newcommand{\SpecialCharTok}[1]{\textcolor[rgb]{0.81,0.36,0.00}{\textbf{#1}}}
\newcommand{\SpecialStringTok}[1]{\textcolor[rgb]{0.31,0.60,0.02}{#1}}
\newcommand{\StringTok}[1]{\textcolor[rgb]{0.31,0.60,0.02}{#1}}
\newcommand{\VariableTok}[1]{\textcolor[rgb]{0.00,0.00,0.00}{#1}}
\newcommand{\VerbatimStringTok}[1]{\textcolor[rgb]{0.31,0.60,0.02}{#1}}
\newcommand{\WarningTok}[1]{\textcolor[rgb]{0.56,0.35,0.01}{\textbf{\textit{#1}}}}
\usepackage{graphicx}
\makeatletter
\def\maxwidth{\ifdim\Gin@nat@width>\linewidth\linewidth\else\Gin@nat@width\fi}
\def\maxheight{\ifdim\Gin@nat@height>\textheight\textheight\else\Gin@nat@height\fi}
\makeatother
% Scale images if necessary, so that they will not overflow the page
% margins by default, and it is still possible to overwrite the defaults
% using explicit options in \includegraphics[width, height, ...]{}
\setkeys{Gin}{width=\maxwidth,height=\maxheight,keepaspectratio}
% Set default figure placement to htbp
\makeatletter
\def\fps@figure{htbp}
\makeatother
\setlength{\emergencystretch}{3em} % prevent overfull lines
\providecommand{\tightlist}{%
  \setlength{\itemsep}{0pt}\setlength{\parskip}{0pt}}
\setcounter{secnumdepth}{5}
\ifLuaTeX
  \usepackage{selnolig}  % disable illegal ligatures
\fi
\usepackage{bookmark}
\IfFileExists{xurl.sty}{\usepackage{xurl}}{} % add URL line breaks if available
\urlstyle{same}
\hypersetup{
  pdftitle={Synoptic CB: Porewater SO4/Cl},
  pdfauthor={2022-2024 Samples},
  hidelinks,
  pdfcreator={LaTeX via pandoc}}

\title{Synoptic CB: Porewater SO4/Cl}
\author{2022-2024 Samples}
\date{2025-10-22}

\begin{document}
\maketitle

{
\setcounter{tocdepth}{2}
\tableofcontents
}
\subsection{Grab 2022 and 2023 All Data
Files}\label{grab-2022-and-2023-all-data-files}

\begin{Shaded}
\begin{Highlighting}[]
\CommentTok{\#read in 2022 Dionex Data:}
\NormalTok{dat22 }\OtherTok{\textless{}{-}} \FunctionTok{read.csv}\NormalTok{(}\StringTok{"2022/COMPASS\_SynopticCB\_SO4\_Cl\_2022.csv"}\NormalTok{)}
\CommentTok{\# head(dat22)}

\CommentTok{\#read in 2023 Dionex Data:}
\NormalTok{dat23 }\OtherTok{\textless{}{-}} \FunctionTok{read.csv}\NormalTok{(}\StringTok{"2023/COMPASS\_SynopticCB\_SO4\_Cl\_2023.csv"}\NormalTok{)}
\CommentTok{\# head(dat23)}

\CommentTok{\#read in 2024 Dionex Data:}
\NormalTok{dat24 }\OtherTok{\textless{}{-}} \FunctionTok{read.csv}\NormalTok{(}\StringTok{"2024/COMPASS\_SynopticCB\_SO4\_Cl\_2024.csv"}\NormalTok{)}
\CommentTok{\# head(dat24)}

\NormalTok{all\_dat }\OtherTok{\textless{}{-}} \FunctionTok{rbind}\NormalTok{(dat22, dat23, dat24)}

\NormalTok{all\_dat }\OtherTok{\textless{}{-}}\NormalTok{ all\_dat }\SpecialCharTok{\%\textgreater{}\%}
  \FunctionTok{select}\NormalTok{(}
\NormalTok{    Project, Region, Sample\_ID, Year, Month, Day, Time, Time\_Zone,}
\NormalTok{    Site, Zone, Replicate, Depth\_cm, }
\NormalTok{    SO4\_Conc\_mM, SO4\_Conc\_flag, SO4\_QAQC\_flag, Cl\_Conc\_mM, Cl\_Conc\_flag, Cl\_QAQC\_flag, salinity,}
\NormalTok{    Analysis\_rundate,  Run\_notes, Field\_notes}
        \CommentTok{\# list columns in the order you want them}
\NormalTok{  )}
\end{Highlighting}
\end{Shaded}

\#\#Make Relevant Metadata Sheet

\subsection{Write out files}\label{write-out-files}

\begin{Shaded}
\begin{Highlighting}[]
\CommentTok{\#write out a csv of all the data to the main folder: }
\FunctionTok{write.csv}\NormalTok{(all\_dat, }\StringTok{"COMPASS\_SynopticCB\_Dionex\_AllData.csv"}\NormalTok{)}


\CommentTok{\#write out a csv of the metadata associated with the data: }
\FunctionTok{write.csv}\NormalTok{(metadat, }\StringTok{"COMPASS\_SynopticCB\_Dionex\_Metadata.csv"}\NormalTok{)}
\end{Highlighting}
\end{Shaded}

\subsection{Visualize Data by Plot}\label{visualize-data-by-plot}

\includegraphics{COMPASS_SynopticCB_Dionex_Collate_AllData_files/figure-latex/Visualize Data-1.pdf}

\subsection{Summarized data for Site and
Zone}\label{summarized-data-for-site-and-zone}

\begin{verbatim}
## `summarise()` has grouped output by 'Month', 'Year', 'Site'. You can override
## using the `.groups` argument.
\end{verbatim}

\includegraphics{COMPASS_SynopticCB_Dionex_Collate_AllData_files/figure-latex/Summarize and Visualize Data-1.pdf}
\includegraphics{COMPASS_SynopticCB_Dionex_Collate_AllData_files/figure-latex/Summarize and Visualize Data-2.pdf}

\subsection{Summarized data for Depth, Site and
Zone}\label{summarized-data-for-depth-site-and-zone}

\begin{verbatim}
## `summarise()` has grouped output by 'Month', 'Year', 'Site', 'Zone'. You can
## override using the `.groups` argument.
\end{verbatim}

\includegraphics{COMPASS_SynopticCB_Dionex_Collate_AllData_files/figure-latex/Depth Data-1.pdf}
\includegraphics{COMPASS_SynopticCB_Dionex_Collate_AllData_files/figure-latex/Depth Data-2.pdf}

\end{document}
